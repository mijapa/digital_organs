%! suppress = Makeatletter
%! suppress = Makeatletter
\documentclass[11pt]{report}

\usepackage[T1]{fontenc}
\usepackage[utf8]{inputenc}
\usepackage{graphicx}
\usepackage{amsmath,amssymb,amsfonts}
\usepackage{polski}
\usepackage[raggedright]{titlesec}
\usepackage{indentfirst}
\usepackage{listings}
\usepackage{hyperref}
\usepackage[backend=biber, bibencoding=utf8, style=ieee, dashed=false, isbn=false, doi=false, sorting=anyvt]{biblatex}
\usepackage{caption}
\captionsetup{%
justification=raggedright,
labelfont=bf,
singlelinecheck=off
}

%\addbibresource{library.bib}
%\addbibresource{NEW.bib}

\pagestyle{headings}

\renewcommand{\chaptername}{Rozdział}
\renewcommand{\contentsname}{Spis treści}
\renewcommand{\figurename}{Rys.}
\renewcommand{\tablename}{Tab.}
\renewcommand{\listfigurename}{Spis rysunków}
\renewcommand{\listtablename}{Spis tabel}
\renewcommand{\bibname}{Bibliografia}

\makeatletter
\renewcommand{\l@section}{\@dottedtocline{1}{1.5em}{2.6em}}
\renewcommand{\l@subsection}{\@dottedtocline{2}{4.0em}{3.6em}}
\renewcommand{\l@subsubsection}{\@dottedtocline{3}{7.4em}{4.5em}}
\makeatother

\begin{document}
    \begin{titlepage}
        \centering
        \center{\scshape Sieciowe Systemy Multimedialne}
        \vspace{0.05\textheight}
        \center{\textbf CYFROWE ORGANY}
        \vspace{0.06\textheight}
        \center{\LARGE\bfseries Wykorzystanie symulatora wirtualnych organów piszczałkowych w celu poprawy brzmienia istniejącego instrumentu cyfrowego}
        \center{Using a virtual pipe organ simulator to improve the sound of an existing digital instrument}
        \vspace{0.5\textheight}
        \center{Michał Patyk}
        \vspace{0.03\textheight}
        \center{Kraków 2021}
    \end{titlepage}

    \tableofcontents


    \chapter{Wstęp}\label{ch:wstęp}
    Motywacją do wykonania projektu jest chęć poprawy brzmienia oraz zwiększenia możliwości istniejącego instrumentu cyfrowego.


    \chapter{Przegląd dziedziny}


    \section{Standard MIDI}


    \section{Organy cyfrowe}


    \section{Wirtualne organy piszczałkowe}
    Na rynku wirtualnych organów piszczałkowych istnieją dwa duzi gracze: Hauptwerk oraz GrandOrgue.

    Hauptwerk\ldots

    GrandOrgue\ldots


    \chapter{Koncepcja}


    \section{Stan istniejący}
    Cyfrowe organy Classic 4500 firmy Viscount.


    \section{Cele}
    Maksymalna poprawa brzmienia organów przy minimalnych nakładach finansowych.
    Zwiększenie możliwości głosowych.


    \chapter{Realizacja}


    \section{Przygotowanie komputera}
    Instalacja systemu operacyjnego Linux.

    \section{Konfiguracja oprogramowania}


    \chapter{Ewaluacja}


    \chapter{Podsumowanie}


    \section{Dalsze kierunki rozwoju}
    Wyposażenia komputera w dysk SSD i większą ilość pamięci RAM.
    Budowa organów hybrydowych - fizyczny subbas 16' - w oparciu o urządzenia midi.

    \inputencoding{utf8}

    \newpage
    \addcontentsline{toc}{chapter}{Bibliografia}

%    TODO remove next line
    \nocite{*}
    \printbibliography[title={Bibliografia}]


\end{document}